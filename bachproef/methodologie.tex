%%=============================================================================
%% Methodologie
%%=============================================================================

\chapter{Methodologie}
\label{ch:methodologie}

%% TODO: Hoe ben je te werk gegaan? Verdeel je onderzoek in grote fasen, en
%% licht in elke fase toe welke stappen je gevolgd hebt. Verantwoord waarom je
%% op deze manier te werk gegaan bent. Je moet kunnen aantonen dat je de best
%% mogelijke manier toegepast hebt om een antwoord te vinden op de
%% onderzoeksvraag.

Growth hacking is geen exacte wetenschap en vaak is het een concept waar vele bedrijven en marketeers onbewust mee bezig zijn. Om onderzoek te doen rond growth hacking voor Kriket, een niet technologische start-up met een fysiek product\footnote{Sinds april 2019 heeft Kriket twee nieuwe producten aangekondigd naast de originele krekelreep. 
\begin{itemize}       
	\item De blauw-groene ``Protein Boost``-krekelreep met dadels, pecannoten en een beetje zeezout. 
	\item De paars-oranje ``Cocoa Chirp``-krekelreep met pistachenoten, chocolade en chiazaden.
\end{itemize}
}, kan men verschillende wegen uitgaan. 

Als men weet hoe andere bedrijven growth hacking aanpakken, bewust of niet, kan men hier uit leren en dit toepassen op Kriket. In deze bachelorproef wordt dat gedaan en gaat men door middel van interviews enkele mensen die al vele jaren ervaring hebben binnen marketing, community management, digitale marketing, performance marketing of natuurlijk ook growth hacking zelf. 

Een tweede optie was enquêtes opsturen en hopen op voldoende antwoord. De reden dat er niet gekozen werd voor enquêtes uitsturen zijn de volgende:
\begin{itemize} 
	\item Grote kans op onvoldoende antwoorden
	\item Niet even veel meerwaarde als een diepgaand interview
\end{itemize}

De voorbereiding van de interviews en de juiste geïnterviewden vinden is in dit geval een grote uitdaging. 

Het belangrijkste bij een goed interview in klassiek: goed kunnen luisteren en de juiste vragen stellen. Op voorhand wordt er dus een lijst met vragen opgesteld, deze vragen kunnen gebruikt worden tijden het interview. Niet alle vragen zullen relevant zijn, maar dat is iets dat tijdens de interviesw duidelijk zal worden.

Naast de voorbereiding van deze vragen is het handig om een soort structuur voor te bereiden voor het interview. Zo kan men er voor zorgen dat de antwoorden verkregen worden op de vragen die het belangrijkste zijn. In het begin zullen er open vragen gesteld worden rond growth hacking, of het al dan niet gekend is bij de geïnterviewde. Afhankelijk van het antwoord zal growth hacking worden uitgelegd met enkele antwoorden. Verder zal er dieper worden ingegaan op growth hacking en andere marketingtechnieken die men gebruikt, die eventueel dicht aanleunen bij growth hacking. Natuurlijk is het ook belangrijk om een goed eind doel te hebben en te weten wanneer de belangrijkste vragen zijn beantwoord en wanneer het gesprek dan tot een conclusie kan komen.

De vragenlijst staat min of meer in volgorde van het moment dat ze gesteld worden in de interviews en deze gaat als volgt:
\begin{itemize} 
	\item Heeft u al van growth hacking gehoord?
	\item Zo ja: hoe past u dit toe binnen uw bedrijf?
	\item Zo nee: past u dit concept toe op een andere manier binnen het bedrijf waar u voor werkt?
	\item Wat zijn uw belangrijkste resources voor de groei van een bedrijf? Steunt u op AdWords, Facebook Ads etc?
	\item Heeft u ervaring met Zero Budget Marketing? Wat is hierbij het belangrijkste onderdeel van marketing?
	\item Hoe ging u in het begin van uw carrière om met (community)-groei van een bedrijf en hoe doet u dat nu? Wat is er veranderd in die tijd? Wat heeft u geleerd?
	\item Waaruit bestaat een ideaal marketing team volgens u? In verhouding: hoeveel copywriters, digital marketeers, programmeurs, creatief ontwerpers, enz.?
	\item Hoe belangrijk is viraal gaan voor een bedrijf? Is dit volgens u waar iedere marketeer, zoals uzelf, naar streeft? Of is het een specifiek doel dat op de juiste moment wordt achtervolgd?
	\item Bij growth hacking spreekt men vaak over 2 à 3 belangrijke onderdelen: Creativiteit, Experimenteren/Analyseren van data en tot slot het Automatiseren en toepassen op technisch vlak. Ook bent u niet bewust bezig met growth hacking, deze onderdelen komen ook aan bod bij traditionele of digitale marketing.
	\begin{enumerate}[label*=\arabic*.]
		\item Creativiteit: Hoe gaat u om met het creatieve proces van marketing? Bijvoorbeeld: Zijn hier brainstorm sessies voorzien? Heeft u creatieve ontwerpers die helpen?
		\item Data: Welke tools gebruikt u om informatie te verzamelen over uw doelpubliek? Welke tool is het belangrijkst en waarom?
		\item Automatiseren: Welke rol speelt IT of het IT-team bij marketing volgens u? 
	\end{enumerate}
	\item In de termen van growth hacking: is er een experiment dat u al lang wil uitvoeren, maar nog geen middelen heeft voor gehad?
	\item Growth hacking is vooral gekend bij online start-ups zoals Airbnb, Hotmail en Dropbox. Veranderd er volgens u veel bij growth hacking wanneer het niet meer gaat om een online bedrijf, maar wel een start-up met een fysiek product, zoals de krekelreep van Kriket?
\end{itemize}

Na een grote zoektocht en met focus op de mensen die het meeste ervaring hebben binnen dit werkveld zijn de geïnterviewden:
\begin{itemize} 
	\item Filip Polfliet: Community manager, Digital Marketeer voor verschillende start-ups, heeft met veel kleine bedrijven gewerkt
	\item Damien Querbes: Head of Marketing van Jaimy, bezit heel veel technische kennis over digital marketing
	\item Vriendin van Damien: Werkt bij Verlinvest en werkt momenteel op een project in de voedingsindustrie en doet daarvoor marketing
	\item Christophe Nottebaert: CEO Jaimy
	\item Yves Simon: Head of Engineering
	\item Thomas hugo: Freelance Illustrator en Concept Artist
\end{itemize}

Combinatie van interviews: Marketing + IT + Creatief 

Na deze interviews wordt er een beknopte analyse uitgevoerd van de huidige marketingsituatie van Kriket. Daarna zal er worden besproken hoe growth hacking eventueel een succesvolle marketingtechniek zou kunnen zijn voor Kriket. Dit zal besproken worden als aanvullen op de huidige marketing die Kriket reeds doet.
