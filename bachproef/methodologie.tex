%%=============================================================================
%% Methodologie
%%=============================================================================

\chapter{Methodologie}
\label{ch:methodologie}

%% TODO: Hoe ben je te werk gegaan? Verdeel je onderzoek in grote fasen, en
%% licht in elke fase toe welke stappen je gevolgd hebt. Verantwoord waarom je
%% op deze manier te werk gegaan bent. Je moet kunnen aantonen dat je de best
%% mogelijke manier toegepast hebt om een antwoord te vinden op de
%% onderzoeksvraag.

Growth hacking is geen exacte wetenschap en vaak is het een concept waar vele bedrijven en marketeers onbewust mee bezig zijn. Om onderzoek te doen rond growth hacking voor Kriket, een niet technologische start-up met een fysiek product\footnote{Sinds april 2019 heeft Kriket twee nieuwe producten aangekondigd naast de originele krekelreep. 
\begin{itemize}       
	\item De blauw-groene ``Protein Boost``-krekelreep met dadels, pecannoten en een beetje zeezout. 
	\item De paars-oranje ``Cocoa Chirp``-krekelreep met pistachenoten, chocolade en chiazaden.
\end{itemize}
}, kan men verschillende wegen uitgaan. 

Als men weet hoe andere bedrijven growth hacking aanpakken, bewust of niet, kan men hier uit leren en dit toepassen op Kriket. In deze bachelorproef wordt dat gedaan en gaat men door middel van interviews enkele mensen die al vele jaren ervaring hebben binnen marketing, community management, digitale marketing, performance marketing of natuurlijk ook growth hacking zelf. 

Een tweede optie was enquêtes opsturen en hopen op voldoende antwoord. De reden dat er niet gekozen werd voor enquêtes uitsturen zijn de volgende:
\begin{itemize} 
	\item Grote kans op onvoldoende antwoorden
	\item Niet even veel meerwaarde als een diepgaand interview
\end{itemize}

De voorbereiding van de interviews en de juiste geïnterviewden vinden is in dit geval een grote uitdaging. 

Het belangrijkste bij een goed interview in klassiek: goed kunnen luisteren en de juiste vragen stellen. Op voorhand wordt er dus een lijst met vragen opgesteld, deze vragen kunnen gebruikt worden tijden het interview. Niet alle vragen zullen relevant zijn, maar dat is iets dat tijdens het interview duidelijk zal worden.

Naast de voorbereiding van deze vragen is het handig om een soort structuur voor te bereiden voor het interview. Zo kan zorgt men er voor dat de antwoorden krijgt op de vragen die men zeker wil stellen. En het is natuurlijk belangrijk om een goed eind doel te hebben en te weten wanneer de belangrijkste vragen zijn beantwoord en wanneer het gesprek dan tot een conclusie kan komen.

De vragenlijst staat min of meer in volgorde van het moment dat ze gesteld worden in de interviews en deze gaat als volgt:
\begin{itemize} 
	\item Heeft u al van growth hacking gehoord?
	\item Zo ja: past u dit toe binnen uw bedrijf, zo ja: hoe? Zo nee: waarom niet?
	\item Zo nee: *interviewer geeft uitleg rond growth hacking*, past u dit concept toe op een andere manier binnen het bedrijf waar u voor werkt?
\end{itemize}

Na een grote zoektocht en met focus op de mensen die het meeste ervaring hebben binnen dit werkveld zijn de geïnterviewden de volgende personen:
\begin{itemize} 
	\item Filip: Community manager, Digital Marketeer voor verschillende start-ups, heeft met veel kleine bedrijven gewerkt
	\item Damien: Head of Marketing voor jaimy (huidige stageplaats), bezit heel veel technische kennis over digital marketing
	\item Vriendin van Damien: Werkt bij Verlinvest en werkt momenteel op een project in de voedingsindustrie en doet daarvoor marketing
\end{itemize}

Na deze interviews wordt er een beknopte analyse uitgevoerd van de huidige marketingsituatie van Kriket. Daarna zal er worden besproken hoe growth hacking eventueel een succesvolle marketingtechniek zou kunnen zijn voor Kriket. Dit zal besproken worden als aanvullen op de huidige marketing die Kriket reeds doet.
