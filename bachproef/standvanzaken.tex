\chapter{Stand van zaken}
\label{ch:stand-van-zaken}

% Tip: Begin elk hoofdstuk met een paragraaf inleiding die beschrijft hoe
% dit hoofdstuk past binnen het geheel van de bachelorproef. Geef in het
% bijzonder aan wat de link is met het vorige en volgende hoofdstuk.

% Pas na deze inleidende paragraaf komt de eerste sectiehoofding.
Om growth hacking te kunnen kaderen moet men begrijpen waar het begrip van komt en hoe het ontstaan is. De bron van growth hacking ligt in bij marketing, digitale marketing, influencer marketing enz. In de volgende paragrafen zullen deze begrippen duidelijk gemaakt worden en op het einde van dit hoofdstuk zal er een concrete definitie gevormd worden over growth hacking.

\section{Wat is marketing?}
\label{sec:marketing}
Vrijwel ieder bedrijf heeft een marketingteam.

\section{Invloed van de digitalisering op marketing}
\label{sec:digitalisering-marketing}
De digitalisering heeft een grote invloed gehad op de manier waarop bedrijven aan marketing doen. Een nieuwe "soort" marketing is ontstaan: digitale marketing. 

Hiervoor bestaan nu ook nieuwe jobs, de all-around digital marketeer kan op z'n eentje de marketing doen van een klein bedrijf. Grote bedrijven hebben vaak een specialist in SEO, die zorgt ervoor dat de website van het bedrijf makkelijk gevonden kan worden via populaire zoekmachines zoals Google en Bing. Naast een SEO specialist bestaat er vaak een community manager die via sociale media contact direct heeft met de klanten. Google en Facebook advertising kunnen door aparte teams onderhouden worden. Verder bestaan er ook "lead generator"-experts, die via verschillende regels de ideale manier kunnen vinden om potentiële klanten binnen te halen.


\section{Nieuwe trends binnen marketing}
\label{sec:nieuwe-trends-marketing}
Nu dat digitale marketing matuur is geworden en dat zo wat alle grote bedrijven aan digitale marketing doen, ontstaan er natuurlijk nieuwe trends. Influencer marketing is hier één van. 

- Nieuwe trend: Influencer marketing (Instagram) (evt. interview met Efluenz)
- Content marketing: ref https://www.frankwatching.com/archive/2014/06/23/de-marketingafdeling-van-2020/

Content marketing is een belangrijk fenomeen in de digitale marketing wereld. Het behalen van een goede score bij zoekmachines ligt voor een groot deel bij het schrijven van goede content. Bij het schrijven van deze 'goede content' moet er steeds rekening gehouden worden met de belangrijke keywords die rond het onderwerp vaak gezocht worden. 

In 2014 schreef Bob Oord van RIFF Content Marketing over de marketingafdeling van 2020. In dit artikel bespreekt hij de verschillende rollen die de marketingafdeling in 2020 zou bevatten. Hij spreekt over heel veel rollen, maar deze kunnen ondergebracht worden onder enkele functies of binnen een klein bedrijf 1 functie: de contentmarketeer. 

\section{Wat is growth hacking?}
\label{sec:wat-is-growth-hacking}
Growth hacking is ook een nieuwe trend binnen marketing die vaak onduidelijk met zich mee brengt. Dus dan komt de vraag: wat is growth hacking nu exact?

Wel: den uitleg

\section{Wat is het verschil met guerrillamarketing?}
\label{sec:verschil-met-guerrillamarketing}

info uit podcast van fizzle.co

 -- -- -- 

Guerillamarketing: old-school
- (Tactiek binnen growth hacking?): vb fysiek op een bepaalde plaats flyers uitdelen of het product verkopen (soort van stunts), graffiti, ...

Growth hacking: nieuwe buzz-worthy term
- best -> Creativiteit, want geen budget, probeer het product "out there" krijgen zonder geld
- worst -> hacking the system, no long-term solution
- don't trust it: internet markety thing
- scalable: get users to get more users. Vb: Groupon. Get people to be obsessed: viral factor. Zorg dat mensen heel tevreden zijn van het product, laat hen het verspreiden
- De aanpak van een ingenieur voor marketing : data-driven (met iteraties, opnieuw en opnieuw) experimenten
- Veel feedback vragen: maak het product samen met de klanten
- Een doel zetten: kleinere doelen hebben een groot effect (mentaal) --> zorgt voor momentum. een S.M.A.R.T. goal
- Welke marketing-kanalen kunnen we gebruiken?
- Vb.: Eerst een audience opbouwen en daarna je brand er aan koppelen (via bvb Instagram). Voorbeeld: @portland en portlandgear.
- Het gaat niet altijd plots en heel snel, vaak moet je bijvoorbeeld er heel veel tijd in steken om een bepaald aantal echte volgers te halen op Instagram. 
- growth hacking = analytical mind, get data, samenwerken met ideeën van marketeers en de confirmeren van goede of slechte ideeën
- Welke data? Wie willen we bereiken? Waar vinden we deze personas? Werkt dit experiment? Welk kanaal gebruiken we best? Komen er meer bezoekers? Meer omzet? Wordt het product aangeraden aan vrienden zoals "hebben jullie hier al van gehoord, WANT x en y"?, enz.


Er is een build-in expiration date:
- vb: FB is nu zo groot en het kost veel meer geld, vroeger was het goed als "growth hack", nu is het duur voor de keywords die je nodig hebt



- Gebruiken Belgische start-ups met een fysiek product Growth hacking? (Literatuurstudie of (sub)onderzoeksvraag?)
