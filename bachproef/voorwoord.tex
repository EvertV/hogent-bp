%%=============================================================================
%% Voorwoord
%%=============================================================================

\chapter*{Woord vooraf}
\label{ch:voorwoord}

%% TODO:
%% Het voorwoord is het enige deel van de bachelorproef waar je vanuit je
%% eigen standpunt (``ik-vorm'') mag schrijven. Je kan hier bv. motiveren
%% waarom jij het onderwerp wil bespreken.
%% Vergeet ook niet te bedanken wie je geholpen/gesteund/... heeft

Deze bachelorproef is geschreven in het kader van mijn hogere studies toegepaste informatica. Allereerst zou ik graag mijn promotor Tom Antjon bedanken voor de goede opvolging en raad die hij mij gegeven heeft bij het schrijven van deze bachelorproef. 

Ik heb mijn co-promotor, Michiel Van Meervenne, oprichter van Kriket leren kennen op een warme zomeravond toen ik samen met hem en mijn collega Filip ging petanquen in Brussel. Michiel vertelde toen over zijn start-up, hij ging er bijna volledig invliegen en zijn huidige full-time job stopzetten. Ik was gegrepen door het verhaal achter Kriket, het is niet zomaar een krekelreep. 

Kriket wil de eerste stap zijn naar het eten van insecten voor Brusselaars, Belgen en verder nog iedereen van West-Europa. Het eten van krekels is de eerste stap in het eten van insecten en het laat je denken aan alle dingen rondom ons die we kunnen eten. Het eten van insecten is niet alleen heel voedzaam, het is ook goed voor het milieu. Krekels zijn bijvoorbeeld heel efficiënt, ze zetten voeder 10 keer efficiënter om in eiwit dan koeien, stoten 60 keer minder broeikasgassen uit en verbruiken 300 keer minder water. \autocite{Kriket2018}

Ik werd verliefd op dit concept van Kriket en wou dus enorm graag deze bachelorproef doen in functie van hen, in de hoop Kriket te kunnen helpen. Graag wil ik Michiel bedanken om mijn co-promotor te zijn en tijd voor mij vrij te maken.

Samen met Michiel en Gauthier zaten we samen om te brainstormen waarover de bachelorproef kon gaan. Dit deden we vanboven in de WTC I toren in Brussel, waar ze even de verbouwingen hadden gestaakt en waar vanboven enkele verdiepingen helemaal leeg, zonder binnenmuren stonden. Hierdoor kreeg je een hele grote open ruimte met zicht over heel Brussel. Wat een fantastische locatie...

Oorspronkelijk was het idee om een referral-systeem op te starten. Een referral-systeem kan bijvoorbeeld het volgende zijn. Klanten worden aangespoord om een link (uniek voor iedere klant, bijvoorbeeld 'kriket.be/?ref=evertv') te delen met vrienden. Wanneer de link gebruikt wordt om krekelrepen te kopen krijgen ze beide (klant en vriend) een beloning, zoals 10\% korting of gratis krekelrepen. Het is een systeem dat vele online bedrijven gebruiken. 

Dit idee deed me meteen denken aan growth hacking. Maar wat is dat juist; growth hacking? Op deze vraag wou ik een antwoord krijgen en via het onderzoeksvoorstel kwam ik te weten dat het heel interessant zou zijn om hier onderzoek rond te doen. Het brengt twee werkvelden samen die ik interessant vind: Marketing en IT.

Zo is deze bachelorproef dus tot stand gekomen, ik wens u veel leesplezier!