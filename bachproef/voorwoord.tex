%%=============================================================================
%% Voorwoord
%%=============================================================================

\chapter*{Woord vooraf}
\label{ch:voorwoord}

Deze bachelorproef is geschreven in het kader van mijn hogere studies toegepaste informatica. Allereerst zou ik graag mijn promotor Tom Antjon willen bedanken voor de goede opvolging en alle raad die hij mij gegeven heeft bij het schrijven van deze bachelorproef. 

Verder wil ik mijn copromotor, Michiel Van Meervenne, de oprichter van Kriket bedanken voor deze samenwerking. We hebben elkaar leren kennen op een warme zomeravond toen ik samen met mijn collega Filip en hem ging petanquen in Brussel. Michiel vertelde toen over zijn start-up, hij ging er toen net volledig voor gaan en zijn huidige fulltime job stopzetten. Ik was meteen geïnteresseerd door het verhaal achter Kriket, want wat het is niet zomaar het produceren van krekelrepen. 

Kriket wil de eerste stap zijn naar het eten van insecten voor Brusselaars, Belgen en iedereen van West-Europa. Het eten van krekels kan de eerste stap zijn in het normaliseren van het eten van insecten. Het doet je denken aan alle andere mogelijke voedingsbronnen rondom ons die we momenteel niet benutten. Insecten eten is niet alleen heel voedzaam, het is ook goed voor het milieu. Krekels zijn bijvoorbeeld heel efficiënt, ze zetten voeder 10 keer efficiënter om in eiwit dan koeien, stoten 60 keer minder broeikasgassen uit en verbruiken 300 keer minder water. \autocite{Kriket2018}

Ik vond dit concept van Kriket fantastisch en wou enorm graag onderzoek voeren in functie van hen, in de hoop Kriket te kunnen helpen. 

Begin november 2018 zat ik samen met Michiel en Gauthier (een collega van Michiel) om te brainstormen over deze bachelorproef. Dit deden we vanboven in de WTC I toren te Brussel, waar ze tijdelijk de verbouwingen hadden gestaakt en waardoor de bovenste verdiepingen helemaal leeg stonden (zonder binnenmuren). Het was een hele grote open ruimte waar verschillende hippe start-up hun plek hadden gevonden met zicht over heel Brussel, echt een fantastische locatie.

Via mail haalde Michiel eerder al het idee aan om een referral-systeem op te starten. Een referral-systeem kan bijvoorbeeld het volgende zijn: Klanten worden aangespoord om een link (uniek voor iedere klant, bijvoorbeeld 'kriket.be/?ref=evertv') te delen met vrienden. Wanneer de link gebruikt wordt om krekelrepen te kopen krijgt men een beloning, zoals 10\% korting of gratis krekelrepen. Het is een systeem dat vele online bedrijven gebruiken. 

Dit idee deed me meteen denken aan growth hacking. Maar wat is dat juist; growth hacking? Op deze vraag wou ik een antwoord krijgen en via het onderzoeksvoorstel kwam ik te weten dat het heel interessant zou zijn om hier onderzoek rond te doen. Het brengt twee werkvelden samen die ik interessant vind: Marketing en IT.

Zo is deze bachelorproef dus tot stand gekomen, ik wens u veel leesplezier!