%%=============================================================================
%% Interviews
%%=============================================================================

\chapter{Interviews}
\label{ch:interviews}

De interviews worden in dit hoofdstuk uitgeschreven en op het einde van ieder interview wordt deze samengevat.

\section{Interview Filip Polfiet}
\label{sec:interview-filip}

Filip is een Digital Marketeer, of Migital Darketeer zoals hij zichzelf noemt, met 3 diploma's. Hij heeft eerst zijn diploma journalistiek behaald, daarna politieke en sociale wetenschappen en als laatste confict en development wat ontwikkelings samenwerking betekent. Na al deze diploma's en de vele jaren studeren heeft hij op zichzelf wat liggen omscholen naar de online webtechnologieën. Dit gaat van Wordpress websites tot de achterliggende architectuur en HTML, CSS, JS, ... en verder naar de SEO, SEA en het opvangen van alle data in Google Analytics en zoveel meer. Nu is Filip freelance consultant voor een digital agency en een start-up van de Belfius Bank. Voordien heeft Filip veel ervaring binnen het vak als digital marketeer, web master, community manager en copywriter. Ook heeft hij een goed inzicht in het creatieve deel van marketing zoals het maken van posters, filmpjes, geluidsbewerking en dergelijke.

\begin{itemize} 
	\item Heeft u al van growth hacking gehoord?
	
Ja. Het is vooral een creatieve marketing strategie die veel impact heeft met weinig budget. Denk aan guerrillamarketing. Jammer genoeg is het vaak geforceerd doordat andere bedrijven hetzelfde proberen als andere bedrijven maar dat werkt niet. Of toch meestal niet

	\item Zo ja: hoe past u dit toe binnen uw bedrijf?
	
Vb: Woody: vrij klassiek bedrijf dat geen super opvallende marketing nodig heeft. Het is natuurlijk ook iets dat niet enkel via Facebook Ads kan (wat Filip doet voor Woody als consultant vanuit het digital agency RVI media uit). Jammer genoeg is het ook iets dat veel bedrijven niet durven. Growth hacking is ook durven, dingen uitproberen, experimenteren en falen. Dat is wat meer (grote) bedrijven moeten durven.
	
	\item Zo nee: past u dit concept toe op een andere manier binnen het bedrijf waar u voor werkt?
	
/

	\item Wat zijn uw belangrijkste resources voor de groei van een bedrijf? Steunt u op AdWords, Facebook Ads etc?
	
Groei is een en-en-verhaal. Je moet meerdere kanalen gebruiken zoals E-mail, Facebook Ads, Instagram Ads (via het FB ads platform), Google ads (wat minder belangrijk is omdat niet per sé creatief is, het werkt eerder ondersteunend). FB ads is een goede tool omdat onderzoek doet blijken dat de gemiddelde belg of europeaan minstens 30 min per dag aan zijn GSM besteed.

	
	\item Hoe ging u in het begin van uw carrière om met (community)-groei van een bedrijf en hoe doet u dat nu? Wat is er veranderd in die tijd? Wat heeft u geleerd?
	
Toen Filip begon bestond er geen Social Media en al zeker geen social media advertisements. De websites bestonden nog maar net en waren geen "big deal". 
Nu is er social media en talloze websites. Men kan spreken van een information overload. Het heeft enorm veel voordelen in verband met marketing zoals de toegang tot meerdere kanalen (Social media, e-mail, ...) en dat het heel meetbaar is.
		
	\item Heeft u ervaring met Zero Budget Marketing? Wat is hierbij het belangrijkste onderdeel van marketing?
	
	Het belangrijkste is om heel creatief te zijn. Je moet alle kanalen gebruiken die beschikbaar zijn. Je moet je verspreiden via al die kanalen, en zo'n dingen zijn zeer moeilijk met weinig budget, maar zeker mogelijk, mits goede brainstorm sessies en een goed idee. 
	
	Uit eigen ervaring bij Plan België: op Pukkelpop wou Plan België een actie doen rond kind huwelijken. Initieel wouden ze flyers uitdelen op het festival, maar Filip wou dit origineler aanpakken. Het moet verder gaan dan enkel flyers uitdelen, daarmee bereik je niet veel bij de doelgroep van 15-25 jarigen. Dus Filip bedacht het idee van een stand waar koppels of vrienden zichzelf konden "uithuwelijken". Hier konden ze een ludieke foto nemen en tijdens het proces of achteraf kregen ze een uitleg dat deze leuke stand eigenlijk wel meer betekenis had dan enkel een toffe foto maken met je vrienden. Er zat veel meer achter en bracht zo aandacht bij de jongeren dat er in vele landen wel effectief een probleem is dat jonge meisjes worden uitgehuwelijkt aan een oudere man waar ze helemaal niet mee willen samen zijn. Filip heeft hierbij de volledige uitwerking verzorgd, van idee tot uitwerking van de stand en bijstand van het IT-team. Want dit maakt het zero budget marketing idee ook wel growth hack, doordat IT, marketing en creatieve ontwerpers samenkomen. Aan deze stand was een hele technische kant verbonden waarbij je je e-mailadres kon invullen en wanneer je dan een foto nam kreeg je die foto opgestuurd naar je e-mailadres. Ook was er een iPad voorzien waarmee je een foto kon nemen en deze eenvoudig uploaden op je social media.
	
	Een tweede ervaring, ook bij Plan België, ging over een simpele social media post dat inspeelde op een in die tijd populaire trend: ``Keep calm and ****``. Filip speelde hier op in tijdens moederdag, waar hij de briefing kreeg om een leuke post te maken voor sociale media in verband met sociale media en moederdag. Hij schreef een eenvoudige post met ``Keep calm and don't forget moederkesdag``. Enkel de manier waarop ``moederkersdag`` werd geschreven zorgde voor verschillende reacties en mensen die reageerde op deze post, elkaar tagde enz... Dit toont aan hoe belangrijke de juiste copywriting is en een combinatie van een originele post die inspeelt op de huidige trends.
	
	Een derde ervaring bij Plan België is een groot IT project waarbij ze een eigen crowd funding platform hebben kunnen starten voor vrij weinig geld, ongeveer 25.000 euro. Hier kon iedereen een eigen project starten met een foto en wat tekst waarbij ze hun reden voor geldinzameling verkondigden. Dit kan bijvoorbeeld een huwelijksjubileum zijn, waar de twee getrouwde geen cadeaus willen, maar enkel een goed doel steunen. Dat kan dan via het online platform waar ze enkel een link moeten delen. Daardoor moet het niet meer via hun eigen bankrekening gebeuren. Deze link kan ook verder op sociale media gedeeld worden en zorgt dus voor het belangrijke onderdeel van marketing: referral.
	
	Het laatste voorbeeld is bij Woody, het pyjama bedrijf in Gent: je moet steeds de realiteit en actualiteit in het oog houden. Je moet als marketeer heel bewust zijn van de feiten en deze in iets positief kunnen omzetten. Bijvoorbeeld: de website van Woody lag tijdens de Sint-periode plat, door een technisch probleem. Als digital marketeer kan je hier heel leuk mee omgaan, zoals Filip deed, met een ludieke post. Filip heet een foto van de sint die iemand voor Woody had gemaakt, bewerkt met een draad, waar de sint over viel. Hierbij werd te tekst geplaatst "De sint struikelde over de draad en heeft perongeluk de website neergehaald, we doen ons best om dit op te lossen!". Door deze leuke post is iedereen op de hoogte en toch niet boos op het bedrijf.
	
	\item Waaruit bestaat een ideaal marketing team volgens u? In verhouding: hoeveel copywriters, digital marketeers, programmeurs, creatief ontwerpers, enz.?
	
	De basis is volgens Filip sowieso: een copywriter, grafisch ontwerper en een programmeur/analytisch denker dit met de cijfers en de code kan werken. Hierbij hoort steeds ook een out-of-the-box denker of een gek persoon. Deze kan 1 van de andere functies delen, zoals copywriter.	
	
	Verschillende personen zijn belangrijk, natuurlijk is dit afhankelijk van de grootte van het bedrijf. Er mag geen negatief persoon tussen zitten die niet open staat voor andere ideeën tijdens de brainstorm sessie, want zo creëer je een negatieve sfeer. Je mag tijdens zo'n sessies niet te realistisch zijn; ``dit kan niet want x en y`` is niet wat je wil. Je kan moeilijk op een goed, creatief idee komen in een vergadering met 20 mensen waarvan enkele accountancy en managers dit te veel met cijfers bezig zijn. Je moet los van de cijfers, los van het budget kunnen ``freewheelen`` en het volledige tegenovergestelde durven denken dan dat wat er gepland is. Dit kan moeilijk om een voorafgedefinïeerde tijdspanne, zoals 4 uur in de namiddag. Zo'n ideeën hebben tijd nodig.
	
	\item Hoe belangrijk is viraal gaan voor een bedrijf? Is dit volgens u waar iedere marketeer, zoals uzelf, naar streeft? Of is het een specifiek doel dat op de juiste moment wordt achtervolgd?
	
Dit gaat natuurlijk niet altijd en is iets dat moeilijk is. Het kan bijvoorbeeld 1 keer per jaar, maar het is niet iets dat consistent kan gebeuren. De post op sociale media moet bijvoorbeeld heel creatief, goed gevonden en origineel zijn voordat deze viraal kan gaan. De creativiteit staat centraal hierbij. Natuurlijk hopen veel digital marketeers dat hun post viraal gaat, maar dit is natuurlijk zelden het geval. Bedrijven moeten hier ook niet op focussen, goede content is beter dan geforceerde content die bedoeld is om viraal te gaan.

Een virale post kan van alles zijn en is vaak toevallig. Het kan een goed getimede gif zijn van een sociale media community manager of een leuke reactie op Twitter of een goede post die op de juiste moment is gepost in de juiste context, ... Soms lijken deze reacties of posts heel spontaan, maar is er door de bedrijven wel goed over nagedacht.
	
	\item Bij growth hacking spreekt men vaak over 2 à 3 belangrijke onderdelen: Creativiteit, Experimenteren/Analyseren van data en tot slot het Automatiseren en toepassen op technisch vlak. Ook bent u niet bewust bezig met growth hacking, deze onderdelen komen ook aan bod bij traditionele of digitale marketing.
	\begin{enumerate}[label*=\arabic*.]
		\item Creativiteit: Hoe gaat u om met het creatieve proces van marketing? Bijvoorbeeld: Zijn hier brainstorm sessies voorzien? Heeft u creatieve ontwerpers die helpen?
		
		Volgens Filip is het de basis van alles. Je moet eerst veel nadenken voordat je begint met ontwikkeling van het technische aspect van je idee. De start is steeds een goed en creatief idee. Het is belangrijk om hier voldoende tijd aan te spenderen voordat je verder gaat.
		
		\item Data: Welke tools gebruikt u om informatie te verzamelen over uw doelpubliek? Welke tool is het belangrijkst en waarom?
		
		Google Analytics
		
		\item Automatiseren: Welke rol speelt IT of het IT-team bij marketing volgens u? 
		
		Het IT team speelt een heel belangrijke rol, maar daar loopt het vaak fout. De communicatie en vlotte werking is cruciaal, het is vaak moeilijk om iets gedaan te krijgen in een IT team. Hiervoor is goede communicatie belangrijk.
		
		De laatste 2 punten, Data en automatiseren zijn bijkomend en eerder logisch, terwijl het eerste punt, de creativiteit het belangrijkste en moeilijkste proces is van de 3.
		
	\end{enumerate}
	\item In de termen van growth hacking: is er een experiment dat u al lang wil uitvoeren, maar nog geen middelen heeft voor gehad?
	
	Een nieuw concept, een eigen start-up waar alles mogelijk is. Niet per sé iets heel concreet, maar een bedrijf waar men durft risico's nemen, want het risico is klein, want het kost niet veel.
	
	\item Growth hacking is vooral gekend bij online start-ups zoals Airbnb, Hotmail en Dropbox. Veranderd er volgens u veel bij growth hacking wanneer het niet meer gaat om een online bedrijf, maar wel een start-up met een fysiek product, zoals de krekelreep van Kriket?
	
	Het is zeker mogelijk, want het is een nieuw, origineel en creatief product. Het belangrijkste wordt een originele invalshoek vinden. Voor Kriket is het belangrijk om in te spelen op de actualiteit, zoals het klimaat. Ze moeten zoeken naar wie er geïnteresseerd is in een Kriket-bar. Wie durft zo'n krekelreep op te eten? Dat moeten ze uitzoeken. Een voorbeeld kan zijn; de ``Kriket challenge``. Hier kan men inspelen op de brug dat de meeste belgen maken van nog nooit insecten gegeten te hebben naar een Kriket-bar. Na de krekelreep is het ijs vaak gebroken en durft men andere insect-gebaseerde gerechten te eten. Rond het concept van de challenge kunnen enkele leuke video's gemaakt worden die in een marketing campagne gebruikt kunnen worden. De video's kunnen de extreme emoties weergeven die mensen hebben bij hun eerste bijt van een krekelreep met achteraf de brede glimlach waaruit blijkt dat het echt lekker is. En dat met geen schrik moest hebben voor de krekels in de krekelreep. ``Durft gij het aan?`` zou een mogelijke manier zijn om de campagne te nemen. Waar willekeurige mensen worden uitgedaagd om de krekelreep te proberen.
	
	\item Wat denk je van growth hacking? 
	
	Goed als klein bedrijf dat zoekt naar snelle groei. Het houdt in dat je veel risico neemt en heel creatief bent met je campagne of growth hack.
	
\end{itemize}


\section{Interview Damien}
\label{sec:interview-damien}

Interview met Damien.
\begin{itemize} 
	\item Heeft u al van growth hacking gehoord?
	

	
	\item Zo ja: hoe past u dit toe binnen uw bedrijf?
	

	
	\item Zo nee: past u dit concept toe op een andere manier binnen het bedrijf waar u voor werkt?
	

	
	\item Wat zijn uw belangrijkste resources voor de groei van een bedrijf? Steunt u op AdWords, Facebook Ads etc?
	


	\item Heeft u ervaring met Zero Budget Marketing? Wat is hierbij het belangrijkste onderdeel van marketing?
	
Heel moeilijk, je product moet heel uniek zijn
	
	\item Hoe ging u in het begin van uw carrière om met (community)-groei van een bedrijf en hoe doet u dat nu? Wat is er veranderd in die tijd? Wat heeft u geleerd?
	

	
	\item Waaruit bestaat een ideaal marketing team volgens u? In verhouding: hoeveel copywriters, digital marketeers, programmeurs, creatief ontwerpers, enz.?
	
	
	
	\item Hoe belangrijk is viraal gaan voor een bedrijf? Is dit volgens u waar iedere marketeer, zoals uzelf, naar streeft? Of is het een specifiek doel dat op de juiste moment wordt achtervolgd?
	
	
	
	\item Bij growth hacking spreekt men vaak over 2 à 3 belangrijke onderdelen: Creativiteit, Experimenteren/Analyseren van data en tot slot het Automatiseren en toepassen op technisch vlak. Ook bent u niet bewust bezig met growth hacking, deze onderdelen komen ook aan bod bij traditionele of digitale marketing.
	\begin{enumerate}[label*=\arabic*.]
		\item Creativiteit: Hoe gaat u om met het creatieve proces van marketing? Bijvoorbeeld: Zijn hier brainstorm sessies voorzien? Heeft u creatieve ontwerpers die helpen?
		
		
		
		\item Data: Welke tools gebruikt u om informatie te verzamelen over uw doelpubliek? Welke tool is het belangrijkst en waarom?
		
		
		
		\item Automatiseren: Welke rol speelt IT of het IT-team bij marketing volgens u? 
		
		
		
	\end{enumerate}
	\item In de termen van growth hacking: is er een experiment dat u al lang wil uitvoeren, maar nog geen middelen heeft voor gehad?
	
	
	
	\item Growth hacking is vooral gekend bij online start-ups zoals Airbnb, Hotmail en Dropbox. Veranderd er volgens u veel bij growth hacking wanneer het niet meer gaat om een online bedrijf, maar wel een start-up met een fysiek product, zoals de krekelreep van Kriket?
	
	
	
\end{itemize}


\section{Interview 3}
\label{sec:interview-3}

Interview 3.

\section{Interview 4}
\label{sec:interview-4}

Interview 4.