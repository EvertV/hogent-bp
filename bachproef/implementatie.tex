%%=============================================================================
%% Analyse marketing Kriket
%%=============================================================================

\chapter{Growth hack toegepast op Kriket}
\label{ch:implementatie}

Het voorbeeld dat in het vorige hoofdstuk aangehaald wordt zal hier worden uitgewerkt. Zo kan men bewijzen dat het toepassen van growth hacking kan op niet-technologische bedrijven mogelijk is. 

Het referral systeem, genaamd Referral Candy, zal in de website van Kriket (\href{https://kriket.be}{kriket.be}) worden geïmplementeerd. Dit gebeurt in samenhang van het lanceren van een nieuw product, de \emph{discover box}.

\section{Stappenplan}
\label{sec:implementatie-stappenplan}

Het stappenplan voor deze specifieke implementatie is als volgt:

\begin{enumerate}
	\item Huidige situatie bekijken
	\item Google Analytics opzetten
	\item Referral Candy toevoegen aan de Shopify website
	\item A/B test opzetten op layout van referral formulier
	\item Growth hack online zetten
	\item Monitoring gedurende twee weken
	\item Data analyseren: A/B test, verkoopcijfers, aantal bezoekers, enz.
	\item Conclusie: growth hack succesvol?
\end{enumerate}

Deze worden stap voor stap uitgeschreven op de volgende pagina's. 

\subsection{Huidige situatie bekijken} \label{sec:huidige-situatie-analyseren}

Hieronder staat een overzicht van de gebruikte technologieën en platformen.

\textbf{Webshop}
\begin{itemize}
	\item Shopify
\end{itemize}

\textbf{Data en analytics}
\begin{itemize}
	\item Shopify Analytics
\end{itemize}

\textbf{Sociale media platformen}
\begin{itemize}
	\item Facebook
	\item Instagram
\end{itemize}

Dit is een goede basis waaraan Google Analytics binnen \emph{Data en analytics} wordt toegevoegd en aan de webshop wordt de module voor het referral systeem toegevoegd.

\subsection{Google Analytics opzetten} \label{sec:google-analytics-opzetten}
Het koppelen van de Shopify website met Google Analytics is een eenvoudig proces. De Google account van Kriket werd gebruikt om een Google Analytics profiel aan te maken. De implementatie gebeurde zo:

\subsection{Referral Candy toevoegen} \label{sec:referral-candy-toevoegen}
De Shopify applicatie Referral Candy wordt toegevoegd en geconfigureerd naar de noden van Kriket. Het volgende concept wordt toegepast:

\emph{Arno} is heel tevreden van Kriket en wil het product aanraden aan zijn vriendin \emph{Bonnie}. Bonnie krijgt volgende URL van Arno opgestuurd via Facebook Messenger ``kriket.be/shop?referral=arno``. Bonnie is geïnteresseerd in het concept van Kriket en koopt via de link van Arno een doos van de discover box. Door de link van Arno te gebruiken heeft Bonnie 20\% kunnen besparen én krijgt ze gratis verzending. Arno krijgt op zijn beurt ook voordeel en ontvangt 20\% korting op zijn volgende 5 aankopen (met hetzelfde e-mailadres).

\subsection{A/B test opzetten op layout van referral formulier} \label{sec:a-b-test-opzetten}
Er worden twee designs getest aan de hand van een A/B-test. De twee mogelijkheden zijn onderaan te zien, versie A: \ref{fig:a-b-test-versie-a} en versie B: \ref{fig:a-b-test-versie-b}. De A/B-test worden uitgevoerd via Google Optimize, deze is op zijn beurt gekoppeld met Google Analytics om de test optimaal op te volgen.

\subsection{Growth hack online zetten} \label{sec:growth-hack-online-zetten}
Zodra het opzetten van alle voorgaande tools getest is kan de growth hack online gezet worden. Hierbij worden de twee sociale media platformen gebruikt. Langs Instagram en Facebook komen de fans en klanten van Kriket te weten dat het referral systeem beschikbaar is. Indien ze de beloning voldoende vinden en dat deze de moeite waard is, zullen ze het wellicht gebruiken. 

\subsection{Monitoring gedurende twee weken} \label{sec:monitoring-gedurende-twee-weken}
Nu dat de growth hack online staat kan Google Analytics gebruikt worden voor de monitoring van deze growth hack. Hieruit zal blijken hoe vaak het referral systeem gebruikt wordt, indien blijkt dat het niet werkt zal men de beloning moeten verhogen of bijsturen op een ander gebied.

\subsection{Data analyseren} \label{sec:data-analyseren}
De data uit Google en Shopify Analytics zegt ons het volgende:

*grafiek*

A/B test, verkoopcijfers, aantal bezoekers, enz.

\subsection{Conclusie: growth hack succesvol?} \label{sec:conclusie-growth-hack-succesvol}
De conclusie