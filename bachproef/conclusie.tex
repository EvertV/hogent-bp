%%=============================================================================
%% Conclusie
%%=============================================================================

\chapter{Conclusie}
\label{ch:conclusie}

%% TODO: Trek een duidelijke conclusie, in de vorm van een antwoord op de
%% onderzoeksvra(a)g(en). Wat was jouw bijdrage aan het onderzoeksdomein en
%% hoe biedt dit meerwaarde aan het vakgebied/doelgroep? Reflecteer kritisch
%% over het resultaat. Had je deze uitkomst verwacht? Zijn er zaken die nog
%% niet duidelijk zijn? Heeft het onderzoek geleid tot nieuwe vragen die
%% uitnodigen tot verder onderzoek?

De interviews en de analyse van het vorige hoofdstuk leren ons veel. We kunnen de informatie van alle vorige hoofdstukken gebruiken om antwoord te vinden op de onderzoeksvraag:

\emph{Kan growth hacking toegepast worden op een start-up met een fysiek product?}

Uit de interviews blijkt dat de geïnterviewden geen niet zomaar denken dat het mogelijk is, maar ongetwijfeld zeker zijn dat Kriket growth hacking kan toepassen. Het is een heel uniek product, wat zeker volgens Damien~(\ref{sec:interview-damien}) de belangrijkste factor is om te groeien. 

Vele ideeën en tips die uit de interviews en het onderzoek steeds terug kwamen zijn zaken die Kriket reeds toe past:
\begin{itemize}
	\item Gratis proevers uitdelen
	\item Interessante partners zoeken om mee samen te werken
	\item Influencer marketing toepassen
	\item Samenwerken met grote supermarkten (Delhaize, Carrefour, ...)
\end{itemize}

Enkele aanvullende wegen die Kriket kan uitgaan:
\begin{itemize}
	\item Volledig digitaal gaan met abonnement-formule, dit gaat verder dan growth hacking alleen
	\begin{itemize}
		\item Voorbeelden zoals \href{https://www.dollarshaveclub.com/}{Dollar Shave Club} of \href{https://www.hellofresh.be/}{HelloFresh} volgen.
		\item Heel hevig inspelen op influencer marketing, niet enkel op instagram, maar ook Youtube, Podcasts, etc.
		\item Robuuste website creëren met mogelijkheid om je abonnement aan te passen naar jouw wensen, op ieder moment van de dag
	\end{itemize}
	\item Een marketing video maken waar verschillende mensen op straat een krekelreep van Kriket proberen. Hierbij wordt de ``shock``-factor goed in beeld gebracht met achteraf het vreugdige beeld dat het gevoel weerspiegeld van een lekkere snack. Dit kan gepubliceerd worden op sociale media, als het goed aangepakt wordt heeft dit potentieel om binnen de juiste doelgroep viraal te gaan.
\end{itemize}

Wat Kriket volgens onderzoek beter vermijd of waar er voldoende over moet nagedacht worden:
\begin{itemize}
	\item Facebook advertenties zijn volgens Thomas~(\ref{sec:interview-thomas-hugo}) niet dé oplossing, wat Thomas beschreef als de oplossing is wat Kriket nu doet. Uitdelen van proevers zodat mensen over die eerste drempel geraken, de drempel is voor de eerste keer insecten eten.
\end{itemize}

Enkele specifieke growth hacks die Kriket zou kunnen toepassen:
\begin{itemize}
	\item Indien de abonnement-formule wordt geïmplementeerd kan er veel men de webshop gespeeld worden, de ``?location=brussel`` tag kan die Damien~(\ref{sec:interview-damien}) vermelde zou dan bijvoorbeeld gebruikt kunnen worden. Dit, maar ook vele andere kleine hacks.
	\item Een referral-systeem waarbij beide partijen baat bij hebben. Wanneer men iemand doorverwijst naar Kriket en die koop een doos krijgt ieder 1 kleine doos gratis.
\end{itemize}

Growth hacking is niet veel voorkomend bij niet-technologische of niet-online bedrijven, maar dit betekend niet dat growth hacking onmogelijk wordt. Aangezien zo goed als alle start-ups (en breder nog, alle bedrijven) online aanwezig zijn is er de mogelijkheid om growth hacking als marketingstrategie toe te passen. 

\section{Mogelijke uitwerking van een experiment}
\label{sec:mogelijke-uitwerking-van-een-experiment}
Stel dat Kriket de growth hack wil proberen waarbij er een referral-systeem wordt opgezet, dan zou het uitgewerkt kunnen worden zoals beschreven in de volgende alinea's.

Hierbij zullen we de ``regels`` van growth hacking volgen, dat wil zeggen dat we gaan kijken naar de implementatie en hoe deze aangepakt wordt. En natuurlijk ook naar het opvangen van de data, het testen, de mogelijke vaststellingen en wat er dan uit geleerd kan worden. Daaruit vloeien dan leerrijke lessen die zorgen voor aanpassingen in het experiment. Zo wordt er gebouwd naar een goede growth hack.



 
