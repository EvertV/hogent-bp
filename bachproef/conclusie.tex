%%=============================================================================
%% Conclusie
%%=============================================================================

\chapter{Conclusie van de studie}
\label{ch:conclusie}

De interviews en de analyse van het vorige hoofdstuk leren ons veel. We kunnen de informatie van alle vorige hoofdstukken gebruiken om antwoord te vinden op de onderzoeksvraag:

\emph{Kan growth hacking toegepast worden op een niet-technologische Brusselse start-up met een fysiek product?}

Uit de interviews blijkt dat de geïnterviewden zeker zijn dat Kriket growth hacking kan toepassen. Het is een heel uniek product, wat zeker volgens Damien~(\ref{sec:interview-damien}) de belangrijkste factor is om te groeien. 

Vele marketing ideeën en tips die uit de literatuurstudie en interviews steeds terug kwamen zijn zaken die Kriket toe past op het moment van publiceren van deze bachelorproef. 
\begin{itemize}
	\item Gratis proevers uitdelen
	\item Interessante partners zoeken om mee samen te werken (zoals Eurowings)
	\item Influencer marketing toepassen
	\item Samenwerken met grote supermarkten (Delhaize, Carrefour, ...)
\end{itemize}

Enkele aanvullende wegen die Kriket kan uitgaan:
\begin{itemize}
	\item Volledig digitaal gaan met abonnement-formule, dit gaat natuurlijk verder dan enkel growth hacking. Het is een aanpassing van het business model.
	\begin{itemize}
		\item Voorbeelden zoals \href{https://www.dollarshaveclub.com/}{Dollar Shave Club} of \href{https://www.hellofresh.be/}{HelloFresh} volgen.
		\item Heel hevig inspelen op influencer marketing, niet enkel op instagram, maar ook met bekende Youtubers, op podcasts, etc.
		\item Robuuste website creëren met mogelijkheid om je abonnement aan te passen naar jouw wensen, op ieder moment van de dag
	\end{itemize}
	\item Een marketing video maken waar verschillende mensen op straat een krekelreep van Kriket proberen. Hierbij wordt de ``shock``-factor goed in beeld gebracht met achteraf het vreugdige beeld dat het gevoel weerspiegeld van een lekkere snack. Dit kan gepubliceerd worden op sociale media, als het goed aangepakt wordt heeft dit potentieel om binnen de juiste doelgroep viraal te gaan.
\end{itemize}

Waar Kriket volgens onderzoek voldoende aandachtig over moet zijn:
\begin{itemize}
	\item Facebook advertenties zijn volgens Thomas~(\ref{sec:interview-thomas-hugo}) niet dé oplossing, wat Thomas beschreef als de oplossing is wat Kriket nu doet. Uitdelen van proevers zodat mensen over die eerste drempel geraken, de drempel is voor de eerste keer insecten eten. Facebook advertenties kunnen dienen als ondersteuning van de huidige acties, maar als voornaamste bron van marketing zal de te kost hoog oplopen voor het bereik dat men wil genereren.
\end{itemize}

Enkele specifieke growth hacks die Kriket zou kunnen toepassen:
\begin{itemize}
	\item Indien de abonnement-formule wordt geïmplementeerd kan er veel met de webshop gespeeld worden, de ``?location=brussel`` tag die Damien~(\ref{sec:interview-damien}) vermelde zou dan bijvoorbeeld gebruikt kunnen worden. Dit, maar ook vele andere kleine hacks.
	\item Een referral-systeem waarbij beide partijen baat bij hebben. Wanneer men iemand doorverwijst naar Kriket en die koop een doos krijgt ieder 1 kleine doos gratis.
\end{itemize}

\section{Mogelijke uitwerking van een experiment}
\label{sec:mogelijke-uitwerking-van-een-experiment}
Stel dat Kriket de growth hack wil proberen waarbij er een referral-systeem wordt opgezet, dan zou het uitgewerkt kunnen worden zoals beschreven in de volgende alinea's.

Hierbij zullen we de ``regels`` van growth hacking volgen, dat wil zeggen dat we gaan kijken naar de implementatie en hoe deze aangepakt wordt. En natuurlijk ook naar het opvangen van de data, het testen, de mogelijke vaststellingen en wat er dan uit geleerd kan worden. Daaruit vloeien dan leerrijke lessen die zorgen voor aanpassingen in het experiment. Zo wordt er gebouwd naar een goede growth hack.

\subsection{Welke growth hack gaat men gebruiken?}
\label{subsec:welke-growth-hack}
Het vinden van de juiste growth hack is niet eenvoudig, er zijn verschillende factoren die bepalen dat deze goed of slecht is. Er moet bekeken worden of het kan werken binnen de huidige markt en dat jouw doelgroep open staat voor wat je gaat proberen. Vaak vertellen growth hackers dat er best niet té veel over nagedacht wordt en dat men gewoon moet proberen. 

In dit voorbeeld, het referral-systeem, is het niet zomaar een kleine growth hack. Het is een groter IT project waar intensief aan gewerkt moet worden. De reden dat dit kan werken is omdat er vele succesvolle voorbeelden bestaan. Wanneer men zich hierop baseert kan men vaststellen dat het de moeite waard is om te proberen.

De manier waarop men deze growth hack toepast en de effectieve waarden zijn zeer belangrijk voor het succes. Wanneer men als klant, bij wijze van spreken, 10 eurocent korting krijgt als beloning wanneer men voor een nieuwe klant zorgt, dan is het de moeite niet waard om het door te vertellen. Als men als klant twee gratis krekelrepen kan krijgen door iemand klant te maken, dan is het ineens wel de moeite. Hier moet dus een doordachte beslissing genomen worden, want het moet uiteindelijk wel geld opbrengen. Het kan deel zijn van de strategie om beide klanten (oud en nieuw) in het begin van de uitvoering van de growth hack (te) veel gratis krekelrepen te geven. Dit zal dan zorgen voor verlies in het begin, maar uiteindelijk zal dit zorgen voor winst doordat zodanig veel mensen het product kennen en kopen. Hierbij zal een grondige analyse kunnen oordelen over de grootte van het risico.

\subsection{Waar begint men met de implementatie?}
\label{subsec:begin-implementatie}
De basis van deze growth hack ligt enerzijds bij het idee en anderzijds bij de implementatie. Dit is het IT gegeven van het verhaal, hierbij zal men iemand met veel technische kennis moeten inhuren om ofwel een bestaand systeem te integreren in de Shopify webshop van Kriket ofwel een nieuw systeem te creëren voor Kriket specifiek.

De kosten van ontwikkeling van een eigen referral systeem lopen snel heel hoog op. Daarom kan het implementeren van een tool zoals Referral Candy\footnote{\href{https://apps.shopify.com/referralcandy}{Referral Candy} is een Shopify applicatie die 50 dollar per maand kost waarmee men het referral systeem in een website kan integreren.} een betere oplossing worden. 

Tijdens de implementatie van de growth hack, waar het IT-team verantwoordelijk is (het IT-team kan ruim gezien worden in deze context, het kan ook één freelancer zijn), kan het marketing-team de doelen gaan vaststellen. Deze moeten S.M.A.R.T. worden opgesteld (zoals vermeld in \ref{sec:hoe-growth-hacking}) en zo kan men later vaststellen of de verwachtingen in lijn liggen met de realiteit. Het is mogelijk om maar één S.M.A.R.T. doel te hebben, maar het wordt aangeraden om er meerdere te stellen op verschillende tijdstippen. Zo kan men bijvoorbeeld iedere twee weken een doel behalen, dit zorgt momentum en geeft een positieve invloed op de sfeer.

Kriket kan voor deze growth hack volgende doelen stellen:
\begin{itemize}
	\item Na één maand zal het referral-systeem gezorgd hebben voor minstens 50 nieuwe klanten en tussen 100 en 250 likes op de Facebook-pagina.
	\item Na anderhalve maand zal het referral-systeem verantwoordelijk zijn voor 100 nieuwe klanten.
\end{itemize}

De growth hack wordt geïmplementeerd door het IT-team, de doelen zijn gesteld door de marketeers, nu moeten alle analytische tools opgezet worden voor correcte opvolging zodra de growth hack gebruikt wordt.

Kriket kan volgende tools gebruiken:
\begin{itemize}
	\item Shopify analytics: aantal verkopen en klanten analyseren.
	\item Google Analytics kan gekoppeld worden met Shopify om meer data over de gebruikers van de website te verkrijgen. Dit gaat minder over effectieve klanten, eerder over potentiële klanten.
	\item Hotjar: zorgt voor meer inzicht over de manier waarop de website gebruikt wordt door middel van heatmaps\footnote{Een heatmap, of warmtebeeld, geeft aan op welke plaatsen in de website veel geklikt wordt.}. Naast de heatmaps biedt Hotjar ook andere functionaliteiten aan zoals het tonen van een feedback-formulier.
	\item Om al deze tools te ``voeden`` met data kan Segment gebruikt worden. Damien (Hoofdstuk~\ref{sec:interview-damien}) is een grote voorstander om deze tool als ruggengraat van het \emph{data management plan} te gebruiken. Zo moet enkel deze tool geïmplementeerd worden in de website, deze zal alle data capteren en transformeren naar de data die verwacht wordt door de andere tools zoals Google Analytics. Op deze manier kunnen er makkelijk nieuwe platformen of tools (zoals Mixpanel, die focust op de \emph{User Journey} van de gebruiker) worden toegevoegd.
\end{itemize}

\subsection{Hoe werkt het verspreiden van deze growth hack?}
\label{subsec:growth-hack-verspreiden}
``Dat gaat vanzelf\text{!}``, is wat men graag zou horen. Dit is natuurlijk niet het geval, een growth hack zoals het referral-systeem moet in gang gezet worden. Dit kan via een leuke post op sociale media, deze aankondiging kan een heel belangrijke factor zijn in het succesverhaal. 

Nadat de aankondiging geplaatst is moet men zeker zijn dat alle analytische tools goed werken, want vanaf dat moment worden ze cruciaal voor het bijsturen van de growth hack.

Indien het concept op punt staat en de beloning voor de klant voldoende groot is, moet er achteraf weinig betaalde reclame rond gemaakt worden. Mond tot mond reclame zal het werk doen in dit geval. 

\subsection{Hoe kan men bijsturen en wanneer doet men dit?}
\label{subsec:growth-hack-bijsturen}
Aan de hand van alle data die gecapteerd wordt kan men vaststellen of er bijgestuurd dient te worden. Indien men niet zeker is of men moet bijsturen kan er gebruik gemaakt worden van A/B testen. Bijvoorbeeld: op de bedankt-pagina van een aankoop moedigt Kriket hun klanten aan om deze aankoop te delen op sociale media. Dit gebeurt door een grappige boodschap die voor iedere klant uniek is (op basis van zijn naam, adres en/of de datum van bestellen). De afbeelding die onderaan de tekst op deze pagina getoond wordt wil men wijziging. Deze wijziging doet men best niet zonder nadenken, het is beter om de data te laten spreken en dit kan blijken uit A/B testen. Versie A van de pagina heeft de oude afbeelding en versie B de nieuwe afbeelding, na enkele weken tijd kan met zien welke afbeelding zorgt voor het meeste aantal gedeelde berichten op sociale media. Hieruit wordt de beslissing genomen om al dan niet versie B te gebruiken.
 
\section{Growth hacking toepassen op een fysiek product}
\label{sec:growth-hacking-mogelijk}
Growth hacking toepassen op een niet technologische Brusselse start-up die een eetbaar product produceert en verdeeld, dat spreekt niet voor zich. 

Het is niet veel voorkomend bij niet-technologische of niet-online bedrijven, maar dit betekent niet dat growth hacking onmogelijk wordt. Dit blijkt uit de literatuurstudie en de interviews met verschillende experts. Zo goed als alle start-ups (en breder nog, alle bedrijven) zijn tegenwoordig online aanwezig. Dit zorgt voor talloze mogelijkheden om growth hacking als marketingstrategie toe te passen. 

Kriket is geen uitzondering op deze conclusie. Het bovenstaande voorbeeld (\ref{sec:mogelijke-uitwerking-van-een-experiment}) toont aan dat er gewerkt kan worden met de huidige middelen mits toevoeging van een nieuw systeem. Dit zal het geval zijn voor de meeste growth hacking die men wil toepassen.

Doordat Kriket zo origineel, nieuw en goed is, kan het toepassen van de juiste growth hack op het juiste moment zorgen voor een enorme groei van de Brusselse start-up.  