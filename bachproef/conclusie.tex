%%=============================================================================
%% Conclusie
%%=============================================================================

\chapter{Conclusie}
\label{ch:conclusie}

%% TODO: Trek een duidelijke conclusie, in de vorm van een antwoord op de
%% onderzoeksvra(a)g(en). Wat was jouw bijdrage aan het onderzoeksdomein en
%% hoe biedt dit meerwaarde aan het vakgebied/doelgroep? Reflecteer kritisch
%% over het resultaat. Had je deze uitkomst verwacht? Zijn er zaken die nog
%% niet duidelijk zijn? Heeft het onderzoek geleid tot nieuwe vragen die
%% uitnodigen tot verder onderzoek?

De interviews en de analyse van het vorige hoofdstuk leren ons veel. We kunnen de informatie van alle vorige hoofdstukken gebruiken om antwoord te vinden op de onderzoeksvraag:

\emph{Kan growth hacking toegepast worden op een start-up met een fysiek product?}

Uit de interviews blijkt dat de geïnterviewden geen niet zomaar denken dat het mogelijk is, maar ongetwijfeld zeker zijn dat Kriket growth hacking kan toepassen. Het is een heel uniek product, wat zeker volgens Damien~(\ref{sec:interview-damien}) de belangrijkste factor is om te groeien. 

Vele ideeën en tips die uit de interviews en het onderzoek steeds terug kwamen zijn zaken die Kriket reeds toe past:
\begin{itemize}
	\item Gratis proevers uitdelen
	\item Interessante partners zoeken om mee samen te werken
	\item Influencer marketing toepassen
	\item Samenwerken met grote supermarkten (Delhaize, Carrefour, ...)
\end{itemize}

Enkele aanvullende wegen die Kriket kan uitgaan:
\begin{itemize}
	\item Volledig digitaal gaan met abonnement-formule, dit gaat verder dan growth hacking alleen
	\begin{itemize}
		\item Voorbeelden zoals \href{https://www.dollarshaveclub.com/}{Dollar Shave Club} of \href{https://www.hellofresh.be/}{HelloFresh} volgen.
		\item Heel hevig inspelen op influencer marketing, niet enkel op instagram, maar ook Youtube, Podcasts, etc.
		\item Robuuste website creëren met mogelijkheid om je abonnement aan te passen naar jouw wensen, op ieder moment van de dag
	\end{itemize}
	\item 
\end{itemize}

Wat Kriket volgens onderzoek beter vermijd of waar er voldoende over moet nagedacht worden:
\begin{itemize}
	\item Facebook advertenties z
	\item 
\end{itemize}

