%%=============================================================================
%% Analyse marketing Kriket
%%=============================================================================

\chapter{Kriket analyse}
\label{ch:analyse}

Nadat Michiel en Anneleen, de oprichters van Kriket, succesvol een crowdfunding\footnote{Via \href{https://www.growfunding.be/nl/bxl/kriket}{growfunding.be} konden Michiel en Anneleen 13.275 euro verzamelen met steun van 254 growfunders.} afsloot op 11/07/2017 was er al een community onstaan. Deze kleine community is ontstaan door mond to mond reclame, wat de beste reclame is.

Het maken van de krekelrepen deden ze zelf voor een heel jaar, terwijl dat de crowdfunding liep om mensen warm te maken voor dit nieuwe concept. Vanaf oktober 2018 begon Kriket met productie, ze lanceerde een webshop en gingen op zoek verschillende lokale verdelers die hun krekelrepen in de winkels wouden leggen. 

\section{Huidige marketingsituatie}
\label{sec:huidige-marketingsituatie}
Om andere mensen mee te krijgen in het verhaal van Kriket (want het is niet zomaar een krekelreep) geven ze geregeld gratis proevers weg bij één van hun verdelers. Dit wordt gepost op sociale media en zo hebben ze meteen enkele fans die dit kunnen doorvertellen. Tezamen met de mensen die dan toevallig in die winkel hun eten komen kopen kunnen ze het verhaal van Kriket vertellen en eventueel enkele van de repen verkopen.

Dit zorgt dan weer voor mond tot mond reclame, want ``ik heb vandaag in de winkel een krekelreep gegeten, da was kei lekker`` wordt thuis doorverteld, of aan vrienden. 

Kriket heeft een duidelijke en goede communicatie op Facebook en Instagram, het leunt nauw aan bij het product dus daardoor past het allemaal mooi samen. Instagram stories worden ook zeer correct gebruikt en geven dus informatie die tijdelijk of heel relevant is op het moment van publicatie. Alle communicatie gebeurd in het Engels, hiermee kan iedereen van hun doelgroep begrijpen wat ze willen delen, zowel de Nederlandstaligen als Franstaligen binnen België en ook alle buitenlanders. De website \href{https://kriket.be}{kriket.be} is wel drietalig, hierop kunnen de krekelrepen gekocht worden samen met enkele goodies zoals een T-shirt, pet of tote bag. 

De webshop wordt via sociale media af en toe vermeld, maar het is niet het voornaamste kanaal dat ze gebruiken om de krekelrepen te verkopen.

Naast de lokale verdelers kon Kriket bij AVEVE, Carrefour en Delhaize een plaatsje vinden. De krekelrepen kregen (gratis) publiciteit op TV en ze stonden op verschillende vernieuwende events over ``The Food of the Future``. Ook staat Michiel met verschillende interviews in kranten en tijdschriften, zoals die van UNIZO, Bloovi, enz. De krekelrepen zijn zelfs te vinden in een automaat op een middelbare school.

Er wordt ook geëxperimenteerd met influencer marketing, dit doet Kriket met twee sportieve dames, Aline en Tiphaine. Ze plaatsen enkele posts op hun Instagram account en delen een kortingscode via Instagram stories.

\begin{itemize} 
	\item Samenwerken met influencers; influencer marketing (Aline en co)
	\item Free samples uitdelen in verschillende verdelers
	\item Aankondiging van 2 nieuwe producten: Actie rond Vlaanderen met groot bord en free samples (guerillamarketing) + opvolging via sociale media
	\item Actieve sociale media accounts
	\item Inspelen op actuele onderwerpen zoals klimaat
	\item Duidelijk verhaal
	\item Partnerships (Lokale verdelers in grote steden, Aveve, Delhaize en recent: Eurowings)
	
	Eurowings is heel goed want dit vakantiegevoel dat mensen hebben op het vliegtuig gaat goed samen met het ``avontuurlijke`` van Kriket. Mensen gaan vaker iets nieuw proberen in zo'n situatie. Dit koppelt een goed gevoel met Kriket, wat ideaal is.
	
	\item Deelnemen aan verschillende wedstrijden, zorgt ook voor ``gratis`` reclame
\end{itemize}