%%=============================================================================
%% Analyse marketing Kriket
%%=============================================================================

\chapter{Kriket analyse}
\label{ch:analyse}

Kriket begon met produceren vanaf oktober, toen hadden ze al een community opgebouwd door middel van crowdfunding en mond tot mond reclame.

\section{Huidige marketingsituatie}
\label{sec:huidige-marketingsituatie}
\begin{itemize} 
	\item Samenwerken met influencers; influencer marketing (Aline en co)
	\item Free samples uitdelen in verschillende verdelers
	\item Aankondiging van 2 nieuwe producten: Actie rond Vlaanderen met groot bord en free samples (guerillamarketing) + opvolging via sociale media
	\item Actieve sociale media accounts
	\item Inspelen op actuele onderwerpen zoals klimaat
	\item Duidelijk verhaal
	\item Partnerships (Lokale verdelers in grote steden, Aveve, Delhaize en recent: Eurowings)
	
	Eurowings is heel goed want dit vakantiegevoel dat mensen hebben op het vliegtuig gaat goed samen met het ``avontuurlijke`` van Kriket. Mensen gaan vaker iets nieuw proberen in zo'n situatie. Dit koppelt een goed gevoel met Kriket, wat ideaal is.
	
	\item Deelnemen aan verschillende wedstrijden, zorgt ook voor ``gratis`` reclame
\end{itemize}