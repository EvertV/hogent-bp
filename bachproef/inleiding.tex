%%=============================================================================
%% Inleiding
%%=============================================================================

\chapter{Inleiding}
\label{ch:inleiding}

Veel nieuwe (potentiële) klanten bereiken met weinig of geen budget, dat is wat men met growth hacking wil bereiken. Verschillend van traditionele marketing omdat men op een creatieve en vaak innovatieve manier veel mensen probeert te bereiken. 

Voor een start-up, zoals Kriket, is dit uiteraard enorm belangrijk. Een product uit de grond schieten is al een immense stap, maar het aan de man brengen is een ander verhaal. Daar komt growth hacking natuurlijk als een engeltje uit de hemel gevallen. 

Voorbeelden zoals Spotify, Dropbox en Hotmail tonen aan dat er 'growth hacks' (een innovatieve manier om veel mensen te bereiken) bestaan voor bedrijven met een online product.

\section{Probleemstelling}
\label{sec:probleemstelling}

Er wordt samengewerkt met Kriket, een Brusselse start-up die de eerste (100\% Belgische) krekelreep produceert en verkoopt. Deze worden verkocht via hun webshop\footnote{zie \href{https://kriket.be}{kriket.be}} en via dealers waarmee ze persoonlijk contact hebben.

Een start-up wil natuurlijk groeien, voor Kriket is dit niet anders en Growth hacking kan hier bij helpen.

\section{Onderzoeksvraag}
\label{sec:onderzoeksvraag}

Hierbij komt de vraag; kan growth hacking toegepast worden op een start-up met een fysiek product?

\section{Onderzoeksdoelstelling}
\label{sec:onderzoeksdoelstelling}

De behaalde resultaten zullen voornamelijk bestaan uit enkele uitgewerkte experimenten. Een experiment is dus een mogelijke ”hack” die zorgt voor een versnelde community-groei specifiek voor Kriket. De enquêtes of interviews van het marktonderzoek kunnen zorgen voor nuttige data die dan in grafieken gebruikt worden. Indien er een experiment wordt uitgevoerd kan alles gemeten worden via Analytics-tools (zoals eerder vermeld, Google en/of Facebook Analytics). Deze statistieken zullen dan veel inzichten brengen over de hoeveelheid nieuwe bezoekers en klanten, maar ook de weg die ze hebben afgelegd naar Kriket en zoveel meer.

Het vinden van growth hacking technieken voor een niet-technologische start-up zal waarschijnlijk minder conventioneel zijn dan voor technologische start-ups. Hierdoor zal er dus minder informatie over te vinden zijn en dat maakt dit natuurlijk een zinvol en uitdagend onderzoek.
Uit een uitgevoerd experiment zal hopelijk geconcludeerd kunnen worden dat er een stijging is in het aantal (potentiële) klanten van Kriket, hetzij via Facebook, de website of een reeds eerder vermeld kanaal. Een groei in de Kriket-community, door dit onderzoek, zou de mooiste conclusie zijn.

\section{Opzet van deze bachelorproef}
\label{sec:opzet-bachelorproef}

% Het is gebruikelijk aan het einde van de inleiding een overzicht te
% geven van de opbouw van de rest van de tekst. Deze sectie bevat al een aanzet
% die je kan aanvullen/aanpassen in functie van je eigen tekst.

De rest van deze bachelorproef is als volgt opgebouwd:

In Hoofdstuk~\ref{ch:stand-van-zaken} wordt een overzicht gegeven van de stand van zaken binnen het onderzoeksdomein, op basis van een literatuurstudie.

In Hoofdstuk~\ref{ch:methodologie} wordt de methodologie toegelicht en worden de gebruikte onderzoekstechnieken besproken om een antwoord te kunnen formuleren op de onderzoeksvragen.

% TODO: Vul hier aan voor je eigen hoofstukken, één of twee zinnen per hoofdstuk

In Hoofdstuk~\ref{ch:conclusie}, tenslotte, wordt de conclusie gegeven en een antwoord geformuleerd op de onderzoeksvragen. Daarbij wordt ook een aanzet gegeven voor toekomstig onderzoek binnen dit domein.

