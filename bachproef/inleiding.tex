%%=============================================================================
%% Inleiding
%%=============================================================================

\chapter{Inleiding}
\label{ch:inleiding}

Veel nieuwe (potentiële) klanten bereiken met weinig of geen budget, dat is wat men met growth hacking wil bereiken. Verschillend van traditionele marketing omdat men op een creatieve en vaak innovatieve manier veel mensen probeert te bereiken. 

Voor een start-up, zoals Kriket, is dit uiteraard interessant. Een goed product creëren is al een immense stap, maar het aan de man brengen is een ander verhaal. Daar komt growth hacking natuurlijk als een engeltje uit de hemel gevallen. 

Voorbeelden zoals Spotify, Dropbox en Hotmail tonen aan dat er ``growth hacks`` (concrete toepassingen van growth hacking, een innovatieve manier om veel mensen te bereiken) bestaan voor bedrijven met een online product. Deze voorbeelden worden verder uitgelegd in het volgende hoofdstuk.

\section{Probleemstelling}
\label{sec:probleemstelling}

Er wordt samengewerkt met Kriket, een Brusselse start-up die de eerste (100\% Belgische) krekelreep produceert en verkoopt. Deze worden verkocht via hun webshop\footnote{zie \href{https://kriket.be}{kriket.be}} en via verdelers waarmee ze persoonlijk contact hebben.

Een start-up wil natuurlijk groeien, voor Kriket is dit niet anders en growth hacking kan hier bij helpen.

\section{Onderzoeksvraag}
\label{sec:onderzoeksvraag}

Hierbij komt de vraag; kan growth hacking toegepast worden op een start-up met een fysiek product?

\section{Onderzoeksdoelstelling}
\label{sec:onderzoeksdoelstelling}

Het vinden van growth hacking technieken voor een niet-technologische start-up zal waarschijnlijk niet zo eenvoudig zijn. Doordat het voor niet-technologische bedrijven met een fysiek product minder conventioneel is dan voor technologische start-ups. Hierdoor zal er dus minder informatie over te vinden zijn en dat maakt dit natuurlijk een zinvol en uitdagend onderzoek. Op basis van interviews met expert wordt er op de literatuurstudie verder gebouwd, hieruit hoopt men dan een zinvolle conclusie te kunnen formuleren voor Kriket; iets waar ze mee aan de slag kunnen.

\section{Opzet van deze bachelorproef}
\label{sec:opzet-bachelorproef}

% Het is gebruikelijk aan het einde van de inleiding een overzicht te
% geven van de opbouw van de rest van de tekst. Deze sectie bevat al een aanzet
% die je kan aanvullen/aanpassen in functie van je eigen tekst.

Het vervolg van deze bachelorproef is als volgt opgebouwd:

In Hoofdstuk~\ref{ch:stand-van-zaken} wordt een overzicht gegeven van de stand van zaken binnen het onderzoeksdomein, op basis van een literatuurstudie.

In Hoofdstuk~\ref{ch:methodologie} wordt de methodologie toegelicht en worden de gebruikte onderzoekstechnieken besproken om een antwoord te kunnen formuleren op de onderzoeksvragen.

In Hoofdstuk~\ref{ch:interviews} worden de interviews uitgeschreven.

In Hoofdstuk~\ref{ch:analyse} wordt een korte analyse gemaakt van de huidige marketingsituatie van Kriket.

In Hoofdstuk~\ref{ch:conclusie}, tenslotte, wordt de conclusie gegeven en een antwoord geformuleerd op de onderzoeksvragen. Daarbij wordt ook een aanzet gegeven voor toekomstig onderzoek binnen dit domein.

