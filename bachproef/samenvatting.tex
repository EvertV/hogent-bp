%%=============================================================================
%% Samenvatting
%%=============================================================================

% TODO: De "abstract" of samenvatting is een kernachtige (~ 1 blz. voor een
% thesis) synthese van het document.
%
% Deze aspecten moeten zeker aan bod komen:
% - Context: waarom is dit werk belangrijk?
% - Nood: waarom moest dit onderzocht worden?
% - Taak: wat heb je precies gedaan?
% - Object: wat staat in dit document geschreven?
% - Resultaat: wat was het resultaat?
% - Conclusie: wat is/zijn de belangrijkste conclusie(s)?
% - Perspectief: blijven er nog vragen open die in de toekomst nog kunnen
%    onderzocht worden? Wat is een mogelijk vervolg voor jouw onderzoek?
%
% LET OP! Een samenvatting is GEEN voorwoord!

%%---------- Samenvatting -----------------------------------------------------

\chapter*{Samenvatting}

Deze bachelorproef werd opgesteld in samenwerking met Kriket, een Brusselse start-up die krekelrepen produceert en verkoopt via lokale handelaars én hun eigen webshop. 

Iedere start-up wil groeien, dit geldt ook voor Kriket, maar vaak hebben start-ups niet het gigantische marketingbudget van hun concurrenten. Dan ontstaat de vraag rond goedkope marketing en snelle groei; is het mogelijk?

Om deze vraag te beantwoorden werd er eerst onderzoek gedaan naar een interessante marketingstrategie. Het resultaat hierbij was ``growth hacking``, waar de werkvelden marketing en IT samenkomen. Het is een term die frequent gebruikt wordt, maar toch is er veel onduidelijkheid rond.

Dit was de eerste belangrijke vraag: onderzoeken wat growth hacking nu écht betekent. Zodra duidelijk was startte het onderzoek naar de toepasbaarheid op Kriket, een niet-technologische Brusselse start-up met een fysiek product. 

Het is niet vanzelfsprekend dat het voor een niet-technologische start-up ook kan werken, aangezien alle voorbeelden van succesvolle growth hacks\footnote{Een growth hack is het toepassen van een truc of techniek. Deze term wordt verder verduidelijkt in de literatuurstudie.} toegepast zijn op start-ups die een webplatform als product hebben. Dit zijn dus online of technologische start-ups.

Om dit te onderzoeken zijn er interviews afgenomen met experts in het werkveld. De geïnterviewden zijn gespecialiseerd in verschillende onderdelen die gebruikt worden bij growth hacking. Dit gaat van het creatieve deel in marketing tot het technisch opvangen en analyseren van alle data. 

Uit de interviews zijn er voor Kriket enkele interessante conclusies te trekken, maar vooral dat alle experts tot de conclusie kwamen dat het mogelijk is om growth hacking toe te passen. De implementatie van een growth hack op Kriket wordt ook aangehaald met een voorbeeld: een referral-systeem. Het voorbeeld wordt uitgewerkt op basis van de literatuurstudie en informatie die verkregen werd tijdens de interviews. 

Dit onderzoek gaat niet zo zeer diep in op technologisch vlak, maar geeft eerder een duidelijk overzicht van wat growth hacking is en hoe het voor een niet-technologische start-up toch relevant kan zijn. 

Wat na dit onderzoek zeker belangrijk zal zijn voor Kriket en alle andere niet-technologische start-ups is de implementatie van een (of meerdere) growth hack(s). Hier zou men onderzoek kunnen voeren rond de invloed van de growth hack op het bedrijf. De analyse kan gemaakt worden op basis van de cijfers uit (bijvoorbeeld) Google Analytics en de verkoopcijfers.