%---------- Inleiding ---------------------------------------------------------

\section{Introductie} % The \section*{} command stops section numbering
\label{sec:introductie}

Er wordt onderzoek gedaan omtrent de meest efficiënte marketing techniek om veel mensen te bereiken, met focus op Growth Hacking. Dit onderzoek wordt gevoerd specifiek voor Kriket, een Brusselse start-up die een krekelreep 100\% op eigen bodem produceert, verdeeld en promoot.

Het idee voor dit onderzoek kwam uit een brainstorm-sessie met Michiel en Gauthuer van Kriket en mezelf in hun (tijdelijk) kantoor te WTC I Brussel. Oorspronkelijk was het idee om een referral-systeem op te starten. 
Een referral-systeem houdt in dat er bijvoorbeeld een link gegenereerd kan worden voor een klant die Kriket aan zijn vrienden wilt aanraden. Op deze manier zorgen klanten voor meer klanten. Een reden dan ze dit doen kan zijn voor korting of gratis krekelrepen. Het is een systeem dat vele online bedrijven gebruiken. Maar is het de beste growth hacking techniek? Dat zal in de bachelorproef onderzocht worden.

Het onderzoek zal steeds specifiek voor Kriket gebeuren, zoals verschillende growth hacking technieken (theoretisch) toepassen. Ook gaat er bekeken worden of growth hacking het volledige (traditionele of digitale) marketing departement binnen een start-up kan vervangen.

Het kan verder ook nuttig zijn voor het onderzoek en voor Kriket om onderzoek te doen rond marketing strategieën die specifiek in de voedselindustrie gebruikt worden, beter nog bij start-ups in die sector.

%---------- Stand van zaken ---------------------------------------------------

\section{State-of-the-art}
\label{sec:state-of-the-art}

Hier beschrijf je de \emph{state-of-the-art} rondom je gekozen onderzoeksdomein. Dit kan bijvoorbeeld een literatuurstudie zijn. Je mag de titel van deze sectie ook aanpassen (literatuurstudie, stand van zaken, enz.). Zijn er al gelijkaardige onderzoeken gevoerd? Wat concluderen ze? Wat is het verschil met jouw onderzoek? Wat is de relevantie met jouw onderzoek?

Verwijs bij elke introductie van een term of bewering over het domein naar de vakliteratuur, bijvoorbeeld~\autocite{Doll1954}! Denk zeker goed na welke werken je refereert en waarom.

% Voor literatuurverwijzingen zijn er twee belangrijke commando's:
% \autocite{KEY} => (Auteur, jaartal) Gebruik dit als de naam van de auteur
%   geen onderdeel is van de zin.
% \textcite{KEY} => Auteur (jaartal)  Gebruik dit als de auteursnaam wel een
%   functie heeft in de zin (bv. ``Uit onderzoek door Doll & Hill (1954) bleek
%   ...'')

Je mag gerust gebruik maken van subsecties in dit onderdeel.

%---------- Methodologie ------------------------------------------------------
\section{Methodologie}
\label{sec:methodologie}

Hier beschrijf je hoe je van plan bent het onderzoek te voeren. Welke onderzoekstechniek ga je toepassen om elk van je onderzoeksvragen te beantwoorden? Gebruik je hiervoor experimenten, vragenlijsten, simulaties? Je beschrijft ook al welke tools je denkt hiervoor te gebruiken of te ontwikkelen.

%---------- Verwachte resultaten ----------------------------------------------
\section{Verwachte resultaten}
\label{sec:verwachte_resultaten}

Hier beschrijf je welke resultaten je verwacht. Als je metingen en simulaties uitvoert, kan je hier al mock-ups maken van de grafieken samen met de verwachte conclusies. Benoem zeker al je assen en de stukken van de grafiek die je gaat gebruiken. Dit zorgt ervoor dat je concreet weet hoe je je data gaat moeten structureren.

%---------- Verwachte conclusies ----------------------------------------------
\section{Verwachte conclusies}
\label{sec:verwachte_conclusies}

Hier beschrijf je wat je verwacht uit je onderzoek, met de motivatie waarom. Het is \textbf{niet} erg indien uit je onderzoek andere resultaten en conclusies vloeien dan dat je hier beschrijft: het is dan juist interessant om te onderzoeken waarom jouw hypothesen niet overeenkomen met de resultaten.

