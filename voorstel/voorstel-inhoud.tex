%---------- Inleiding ---------------------------------------------------------

\section{Introductie} % The \section*{} command stops section numbering
\label{sec:introductie}

Er wordt onderzoek gedaan omtrent de meest efficiënte marketing techniek om veel mensen te bereiken, met focus op Growth Hacking. Dit onderzoek wordt gevoerd specifiek voor Kriket, een Brusselse start-up die een krekelreep 100\% op eigen bodem produceert, promoot en verdeeld.

Het idee voor dit onderzoek kwam uit een brainstorm-sessie met Michiel en Gauthier van Kriket en mezelf in hun hipster-kantoor te WTC I Brussel. Oorspronkelijk was het idee om een referral-systeem op te starten. 

Een referral-systeem houdt in dat er bijvoorbeeld een link gegenereerd kan worden voor een klant die Kriket aan zijn vrienden wilt aanraden. Op deze manier zorgen klanten voor meer klanten. Ze worden aangespoord om deze link te delen door een beloning, voor beide partijen, zoals 10\% korting of gratis krekelrepen. Het is een systeem dat vele online bedrijven gebruiken. Maar is het de beste growth hacking techniek? Daar zal deze bachelorproef meer inzicht over geven.

Het onderzoek zal steeds specifiek voor Kriket gebeuren, zoals verschillende growth hacking technieken (theoretisch) toepassen. Ook gaat er bekeken worden of growth hacking het volledige (traditionele of digitale) marketing departement binnen een start-up kan vervangen.

Het kan verder ook nuttig zijn voor het onderzoek en voor Kriket om onderzoek te doen rond marketing strategieën die specifiek in de voedselindustrie gebruikt worden, beter nog bij start-ups in die sector.

Verder kan er focus gelegd worden op wélke growth hack Kriket kan gebruiken. De meeste growth hacking voorbeelden komen uit online start-ups zoals Airbnb, Dropbox, Spotify, Hotmail, ... Daarom zal er onderzoek gedaan worden of dit ook toepasbaar is op een Brusselse start-up met een fysiek product.

%---------- Stand van zaken ---------------------------------------------------

\section{State-of-the-art}
\label{sec:state-of-the-art}

Growth hacking is een 'hot topic', er wordt veel over gepraat, maar toch heerst er veel onduidelijkheid over wat het exact is. 

Het is een onderwerp dat vaak gekoppeld wordt met start-ups, omdat het daarbij het meest relevant is. 

Het onderzoek van~\textcite{Lee2016} is een goede inspiratie voor dit onderzoek, er zal veel naar verwezen worden omdat het een goede basis is met degelijke bronnen. Het is belangrijk te vermelden dat dit onderzoek geen kopie wordt van de bachelor thesis van~\textcite{Lee2016}. 

Het toepassen van growth hacking technieken op een Brusselse start-up, die geen technologiebedrijf is, zorgt (vermoedelijk) voor een volledig andere aanpak. 

Verder is er ook onderzoek gedaan rond het verstorende aspect van growth hacking in de master thesis van~\textcite{Bergendal2017}, hier zal rekening mee gehouden worden tijdens het voeren van het onderzoek in functie van Kriket.

Een tweede gelijkaardig onderzoek van~\textcite{Vunk2017} bespreekt growth hacking technieken die gebruikt worden bij start-ups in Estland. Dit kan helpen met conclusies te trekken omtrent plaatselijke factoren die growth hacking technieken beïnvloeden. Zo kunnen er en verschillen vast gelegd worden met de Belgische markt en globale conclusies getrokken worden.

Dit onderzoek zal dus gericht zijn naar Kriket en hoe zij met weinig middelen en onconventionele technieken veel mensen kunnen bereiken.


% Voor literatuurverwijzingen zijn er twee belangrijke commando's:
% \autocite{KEY} => (Auteur, jaartal) Gebruik dit als de naam van de auteur
%   geen onderdeel is van de zin.
% \textcite{KEY} => Auteur (jaartal)  Gebruik dit als de auteursnaam wel een
%   functie heeft in de zin (bv. ``Uit onderzoek door Doll & Hill (1954) bleek
%   ...'')

%---------- Methodologie ------------------------------------------------------
\section{Methodologie}
\label{sec:methodologie}

Een growth hacker is half marketeer en half ingenieur, zoals vermeld in onderzoek van ~\cite{Lee2016}. Men moet weten wat er technisch mogelijk is en die kennis in het achterhoofd houden wanneer er over het marketinggedeelte wordt nagedacht. 

Er zijn overigens geen vaste richtlijnen voor growth hacking, het is een opeenvolging van experimenten die gevoerd wordt om tot de ideale "hack" te komen. Men moet natuurlijk eerst brainstromen over verschillende mogelijke experimenten. Dat is wat in dit onderzoek zal gebeuren: experimenten bedenken die Kriket kunnen helpen met een versnelde groei.

Om deze experimenten te bedenken zal er eerst een klein marktonderzoek nodig zijn, eventueel door enquêtes of interviews bij andere start-ups en kleine voedselproductie bedrijven. 


%---------- Verwachte resultaten ----------------------------------------------
\section{Verwachte resultaten}
\label{sec:verwachte_resultaten}

De behaalde resultaten zullen voornamelijk bestaan uit enkele uitgewerkte experimenten. Een experiment is in deze context dus een mogelijke "hack" die zorgt voor een versnelde community-groei specifiek voor Kriket. 

Indien mogelijk zou het interessant zijn om een experiment uit te voeren en de groei te meten via Google en/of Facebook Analytics. Echter zullen de meeste experimenten theoretisch blijven en zullen er niet rechtstreeks statistieken kunnen uit opgemaakt worden.

De enquêtes of interviews kunnen wel zorgen voor statistieken die in grafieken gebruikt worden.

%---------- Verwachte conclusies ----------------------------------------------
\section{Verwachte conclusies}
\label{sec:verwachte_conclusies}

Er wordt verwacht dat het vinden van growth hacking technieken voor een niet-technologische start-up minder conventioneel is dan voor technologische start-ups. Er zal dus minder informatie over te vinden zijn en dat maakt dit natuurlijk een zinvol onderzoek.

Indien er een experiment wordt uitgevoerd en de gegevens uit Analytics kunnen halen zal er hopelijk geconcludeerd kunnen worden dat er een stijging is in het aantal bezoekers van de Facebook-pagina of website van Kriket.

De groei van de Kriket-community zou de mooiste conclusie zijn.

